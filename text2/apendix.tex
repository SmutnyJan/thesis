\chapter{Seznam použitých zkratek}
\renewcommand{\glossarysection}[2][]{}
\printglossary[type=acronym]



% \chapter{UML diagramy}
% \textbf{\large Tato příloha není povinná a zřejmě se neobjeví v každé práci. Máte-li ale větší množství podobných diagramů popisujících systém, není nutné všechny umísťovat do hlavního textu, zvláště pokud by to snižovalo jeho čitelnost.}



\chapter{Instalační a uživatelská příručka}
\section{Serverová aplikace}
\subsection{Nároky aplikace}
\begin{itemize}
 \item Platforma Node.js ve verzi 0.6.9 nebo novější.
 \item Databáze MongoDB ve verzi 1.8.2 nebo novější.
 \item Použité knihovny (viz část \ref{sec:server:libs} na straně \pageref{sec:server:libs}), zejména rdfstore-js.
\end{itemize}

\subsection{Instalace aplikace}
Adresářovou strukturu aplikace vložte na požadovanou pozici. Do cronu vložte požadovaná pravidla pro obsluhu zdrojů informací (viz část \ref{sec:server:scheduler} na straně \pageref{sec:server:scheduler}). Systém je připravený k~použití.

\subsection{Spuštění aplikace}
\begin{itemize}
 \item Nejprve spusťte databázi: \texttt{mongod \&}
 \item Nyní spusťte aplikaci: \texttt{node index.js \&}
 \item \glsname{SPARQL} endpoint spustíte příkazem: \\ \texttt{\textasciitilde /node\_modules/rdfstore/bin/rdfstorejs webserver \linebreak -{}-store-name pruvodcefitcvut -{}-store-engine mongodb -p 8080}
\end{itemize}

\subsection{Obsluha aplikace}
\begin{itemize}
 \item Zpracování zdroje: \texttt{curl localhost:1337/<název\_zdroje>}
 \item Test \glsname{SPARQL} endpointu: \texttt{curl -v localhost:8080/sparql \\ -{}-data-urlencode "query=select * \{ ?s ?p ?o \} limit 3"}
\end{itemize}

\section{Mobilní aplikace}
\subsection{Nároky aplikace}
Pro běh mobilní aplikace je třeba webový prohlížeč podporující technologie spjaté s HTML5 (JavaScript, CSS, SVG\dots).

\subsection{Instalace aplikace}
Adresářovou strukturu aplikace vložte na požadovanou pozici. Aplikace je připravena k~použití.

\subsection{Spuštění aplikace}
Ve webovém prohlížeči otevřete soubor \texttt{index.html} z adresáře s aplikací.

\subsection{Obsluha aplikace}
Podrobná nápověda je uvedena přímo v aplikaci, doporučuji si ji přečíst.

\section{Proxy mobilní aplikace}
\subsection{Nároky aplikace}
Pro běh proxy mobilní aplikace je třeba webový server s \glsname{PHP} verze 5.

\subsection{Instalace aplikace}
Adresářovou strukturu aplikace vložte na požadovanou pozici. Aplikace je připravena k~použití.




\chapter{Obsah přiloženého CD}

Přiložené CD je strukturováno:

\dirtree{%
.1 /.
.2 index.html\DTcomment{popis obsahu CD}.
.2 text\DTcomment{text diplomové práce}.
.3 DP\_Molnár\_Jan\_2012.pdf\DTcomment{text diplomové práce ve formátu PDF}.
.2 mobil\DTcomment{zdrojové kódy mobilní aplikace}.
.3 index.html\DTcomment{spouštěcí soubor mobilní aplikace}.
.2 proxy\DTcomment{zdrojové kódy proxy mobilní aplikace}.
.3 index.php\DTcomment{spouštěcí soubor proxy mobilní aplikace}.
.2 server\DTcomment{zdrojové kódy serverové aplikace}.
.3 index.js\DTcomment{spouštěcí soubor serverové aplikace}.
.2 navrh\DTcomment{dokumenty vzniklé při návrhu}.
.3 diagramy\DTcomment{diagramy aplikací}.
.3 prototypy\DTcomment{prototypy mobilní aplikace}.
.2 testovani\DTcomment{dokumenty vzniklé při testování}.
.3 vykonnostni\DTcomment{dokumenty vzniklé při výkonnostním testování}.
.2 podklady\DTcomment{podklady diplomové práce}.
.3 mapy\DTcomment{mapy vytvořené pro potřeby práce}.
.3 ontologie\DTcomment{použité ontologie}.
.2 jine\DTcomment{další soubory spjaté s prací}.
.3 obrazovky\DTcomment{ofocené obrazovky mobilní aplikace}.
.3 pruvodce CVUT FEL\DTcomment{aplikace Průvodce ČVUT FEL}.
}
