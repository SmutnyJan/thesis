\chapter{Seznam použitých zkratek}
\renewcommand{\glossarysection}[2][]{}
\printglossary[type=acronym]



\chapter{Instalační a uživatelská příručka}
\todo{spuštění - příkazy}
\section{Serverová aplikace}

\section{Mobilní aplikace}
% \section{Nároky aplikace}
% Aplikace je plně funkční v každém webovém prohlížeči dodržujícím specifikace \emph{XHTML Mobile Profile 1.1} \cite{XhtmlMpDoc}, \emph{ECMAScript Mobile Profile 1.0} \cite{EsMpDoc}, \emph{CSS 2.1} \cite{CssDoc} a \emph{PNG W3C/ISO/IEC version} \cite{PngDoc}, přesto je ale pravděpodobné, že bude fungovat, byť třeba omezeně, i v jiných prohlížečích.
% 
% \section{Instalace aplikace}
% Celý adresář s aplikací nahrajte do požadovaného umístění.\footnote{Při volbě umístění berte ohled na prostředí, do kterého aplikaci nahráváte. Mnoho mobilních telefonů má pro HTML stránky, které aplikaci tvoří, vyhrazené umístění.} Tím je aplikace nainstalována a připravena k použití.
% 
% \section{Spuštění aplikace}
% Aplikaci spustíte otevřením souboru \emph{index.html}, který se nachází v adresáři s aplikací, prostřednictvím internetového prohlížeče.\footnote{V mobilním prostředí například mobilní verze prohlížečů Opera, NetFront, Safari, Nokia Mini Map, Firefox, Internet Explorer...}\footnote{Pokud používáte prohlížeč Opera Mobile a nemáte v něm dialog pro otevření lokálního souboru, zadejte do adresního řádku protokol \emph{file://}, potvrďte a dialog se zobrazí.}
% 
% \section{Možnosti vyhledávání}
% Hledanou místnost můžete zadat do vyhledávacího pole aplikace jako:
% \begin{itemize}
%  \item jméno místnosti (např. KN:E-107),
%  \item jméno místnosti před přečíslováním v roce 2008 (např. K1),
%  \item zažité pojmenování místnosti (např. Zengerova posluchárna),
%  \item jméno sídlícího vyučujícího (např. Josef Vomáčka).
% \end{itemize}
% Zadat lze i část názvu hledaného umístění, při vyhledávání se nerozlišuje velikost písmen a~lze použít ECMAScriptové regulární výrazy, k nimž je návod v aplikaci a specifikace na~\cite{EsReDoc}.
% 
% \subsection{Hledání v okolí}
% Aplikace umožňuje vyhledávání bodů zájmu (občerstvení, výtahy, toalety...) v okolí místnosti. Vyhledejte běžným způsobem místnost a na plánku se v jejím okolí mimo ní zobrazí i~body zájmu.
% 
% \subsection*{Přímo v aplikaci je k dispozici nápověda.}



\chapter{Obsah přiloženého CD}

Přiložené CD je strukturováno:

\dirtree{%
.1 /.
.2 index.html\DTcomment{popis obsahu CD}.
.2 text\DTcomment{text diplomové práce}.
.3 DP\_Molnár\_Jan\_2012.pdf\DTcomment{text diplomové práce ve formátu PDF}.
.2 mobil\DTcomment{zdrojové kódy mobilní aplikace}.
.3 index.html\DTcomment{spouštěcí soubor mobilní aplikace}.
.2 server\DTcomment{zdrojové kódy serverové aplikace}.
.3 index.js\DTcomment{spouštěcí soubor serverové aplikace}.
.2 navrh\DTcomment{materiály vzniklé při návrhu}.
.3 diagramy\DTcomment{diagramy obou aplikací}.
.3 prototypy\DTcomment{prototypy mobilní aplikace}.
.2 podklady\DTcomment{podklady diplomové práce}.
.2 jine\DTcomment{další soubory spjaté s prací}.
.3 obrazovky\DTcomment{ofocené obrazovky mobilní aplikace}.
.3 pruvodce CVUT FEL\DTcomment{aplikace Průvodce ČVUT FEL}.
}
