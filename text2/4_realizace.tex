% (f) Popis implementace/realizace produktu se zaměřením na nestandardní části řešení.

\chapter{Realizace}
% Popis implementace/realizace se zaměřením na nestandardní části řešení.
% Architektonické obrázky. (class diagram asi ne)
Během implementace se objevila celá řada zádrhelů, z nichž většina souvisela s doposud ne zcela zdokumentovanými a odlazenými technologiemi, přesto byla velmi přínosná -- neustálým hledáním různých řešení a alternativ jsem nutně narazil na mnoho dalších zajímavých postřehů a nápadů. Kapitola se zabývá strukturou aplikací, řešenými problémy, volbou použitých technologií, použitými knihovnami, nádtroji a potenciální aplikací v ideálním světě.

\section{Serverová aplikace}
Aplikace bě

\subsection{Řešené problémy}
\todo{Optimalizace, komplikace...}


% \subsection{Ideální implementace}

\subsection{Implementace bezpečnosti}


\section{Mobilní aplikace}
Pod pojmem \textit{mobilní aplikace} se skrývá aplikace pevně nezávislá na výše uvedené \textit{serverové}. Jejím hlavním cílem je sloužit studentovi \glsname{FIT} \glsname{CVUT} jako průvodce, zároveň má ale plnit i účel ukázky využití \glsname{SPARQL} endpointu vytvořeného v \textit{serverové} části.


\subsection{Řešené problémy}

\subsubsection{Změna velikosti displeje}
chrome css v px

\subsubsection{Chyby prohlížečů v implementaci specifikací}
Při práci jsem několikrát narazil na situaci, kdy vše v jednom prohlížeči fungovalo, ve druhém nikoliv, a přesto bylo vše implementované správně podle specifikace. V takovém případě pak bývala chyba na straně prohlížeče, který špatně implementoval určitou specifikaci. Situace nastávala převážně u hodně specifických případů. Příkladem může být chybějící podpora pro nastavení transformace \glsname{SVG} elementu JavaScriptem v prohlížeči Google Chrome \cite{BugChromeTran}, tento problém ale lze obejít vlastním sestavením hodnoty celého atributu. 

\subsubsection{Rozdílné DOM u HTML a SVG}
\glsname{SVG} je, co se týče návrhu \glsname{DOM}, komplikovanější, než HTML -- zatímco u HTML stačí pro přístup k hodnotě atributu 

\subsubsection{Zpracování Cross-Origin zdrojů}
\label{sec:cross-origin}



\subsection{Implementace bezpečnosti}
\todo{Proxy. jQuery...}



\subsection{Ideální implementace}
Finální implementace se od té ideální, co se týče omezení vzniklých neideálním světem, moc neliší. Vzhledem k cílení aplikace na současná nová zařízení bylo možné realizovat téměř vše, co bylo naplánováno -- v uplynulých letech bylo odvedeno mnoho práce na jednotných standardech napříč zařízeními, a co víc, tyto standardy byly implementovány a z velké části se i dodržují. Problémy se proto vyskytly spíše na poli stále ještě v některých ohledech omezených možnostech mobilních zařízení.

\subsubsection{Malá velikost displeje}
Ačkoliv je toto omezení u mobilních zařízení zřejmé, nic to nemění na faktu, že se s ním musí neustále počítat a je třeba vybírat vhodné komponenty a ty vhodně rozvrhovat. Mým cílem proto bylo vizualizovat jen to nejnutnější.

\subsubsection{Různé typy polohovacích zařízení}
Různá zařízení poskytují různé ovládací prvky využívající různé principy, aplikaci je ale nutné přizpůsobit pro všechny najednou. Jiné je ovládání pro myš (touchpad, trackpad), jiné pro klávesy a jiné pro dotyková zařízení. I přes malý displej je třeba u mobilních dotykových zařízení vytvářet ovládací prvky natolik veliké, aby bylo možné je používat prsty.

\subsubsection{Optimalizace rozvržení prvků}
Mobilní zařízení nabízejí pro lepší využití malých displejů mnoho různých otimalizací rozvržení zobrazovaných prvků. Zpravidla jsou tyto vlastnosti velmi užitečné -- komprimují nevyužité místo, ve specifických případech, jako je třeba přesné pozicování ukazatele, ale může docházet k obtížně řešitelným situacím. Jediné místo aplikace, které se bez přesného rozmístění neobejde, jsou mapy. Problém jsem se nakonec rozhodl řešit využitím  \glsname{SVG} mapových podkladů, ve kterých prohlížeče, zdá se, zatím rozvržení neoptimalizují.

\subsubsection{Implementace SVG}
Implementace \glsname{SVG} je v desktopových prohlížečích na velmi dobré úrovni již nějakou dobu,\footnote{Posledním významným desktopovým prohlížečem podporujícím \glsname{SVG} se stal s dlouhým odstupem verzí 9.0 Internet Explorer \cite{CanIUse}, předtím pro něj ale byly dostupné vhodné pluginy.} v nedávné minulosti se ale podpora značně zlepšila i na poli mobilních zařízení \cite{CanIUse}.

U mobilních zařízení nepřekvapuje chybějící implementace některých neklíčových možností \glsname{SVG}, jako je třeba průhlednost v případě Opery Mobile. V této oblasti v současné době rozsáhlému nasazení nepřeje \glsname{SVG} nepodporující hojně využívaný nativní webový prohlížeč Androidu verze 2, od verze 3 je podpora zavedena. Problém lze obejít využitím jiného dostupného prohlížeče.

Překvapením byla pomalá, ačkoliv jinak velmi rozsáhlá, implementace \glsname{SVG} v prohlížeči Mozilla Firefox -- po některých operacích, v mém případě to byla třeba změna hodnot atributu \texttt{viewBox} \glsname{SVG} elementu (posunutí mapy), prohlížeč vyžaduje překreslení celého obrázku, což je velmi zdlouhavé a zabraňuje to plynulému posunu mapy z aktuální na novou pozici, byť by to bylo vhodné pro lepší uživatelovo zorientování se. Tento nedostatek je známý již delší dobu \cite{Bugzilla}.

\subsubsection{Absence popisků}
Velmi užitečnou vlastností dostupnou u desktopových aplikací jsou popisky zobrazující se po najetí myši nad podporovaný prvek. Dá se tímto způsobem velmi dobře implementovat kontextová nápověda, která patří mezi nejpodstatnější zdroje pomoci uživateli. U dotykových mobilních zařízení ale bohužel popisky zpravidla zobrazovat nejdou -- najetím, tedy dotykem prstu, se rovnou vyvolá akce prvku, aniž by se popisek mohl zobrazit.

\subsubsection{Dynamické načítání lokálních souborů}
Z bezpečnostních důvodů došlo v minulosti v prohlížečích k zamezení načítání JavaScriptových souborů z jiných zdrojů, než ze kterých stránka pochází. To v některých případech\footnote{Příkladem je Google Chrome.} přerostlo v zamezení dynamického načítání jakýchkoliv lokálních souborů -- jejich zdroj nemá žádnou hodnotu (\texttt{null}), takže je nelze načíst. Aby byla zachována možnost offline využití aplikace na větší skupině zařízení, byla aplikace přepsána na jediné možné řešení -- explicitní uvedení všech potenciálně vkládaných souborů, což značně omezilo modularitu aplikace.

\subsubsection{Otevírání lokálních souborů}
Omezení zmíněná v předchozím odstavci stupňuje až do extrémů mobilní prohlížeč Chrome Beta, který nepovolí ani přímé otevření lokálního souboru. \todo{Ověřit, možná nefunguje celé Chrome\dots}