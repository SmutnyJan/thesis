% (f) Popis implementace/realizace produktu se zaměřením na nestandardní části řešení.

\chapter{Realizace}
% Popis implementace/realizace se zaměřením na nestandardní části řešení.
Během implementace se objevila celá řada zádrhelů, z nichž mnoho souviselo s doposud ne zcela zdokumentovanými a odlazenými technologiemi, přesto pro mne byla realizace velmi přínosná -- neustálým hledáním různých řešení a alternativ jsem nutně narazil na mnoho dalších zajímavých postřehů a nápadů. Kapitola se zabývá řešenými problémy, implementací bezpečnosti, komplikacemi a potenciální aplikací v ideálním světě.

\section{Serverová aplikace}
Když se zpětně ohlížím za implementací serverové aplikace, náročné to objektivně vůbec nebylo, osobně jsem ale žádnou z používaných technologií / knihoven\dots v té době neznal, takže mi náročnost přišla enormě vysoká. Rozsah popisu realizace serverové aplikace je tak krátký -- mnoho objektivně zajímavých míst se v ní nevyskytlo.

\subsection{Řešené problémy}
V době, kdy byl vytvořený návrh, bylo vytvoření aplikace už poměrně snadné. Popíšu tedy jen ve stručnosti několik základních částí řešení.

\subsubsection{Získání zdroje}


\subsubsection{Zdroj uživatelských jmen}

\subsubsection{Routování}
Aplikace je serverem\footnote{Pozor, zde je důležité vzít vpotaz vymezení pojmů z poznámky \ref{fnt:aplikace} na straně \pageref{fnt:aplikace} -- aplikace, o které je řeč, je \textit{serverovou aplikací} v užším slova smyslu -- není nazývána \textit{serverovou aplikací} proto, že by nutně musela být serverem, ale proto, že je jednou z komponent serverového řešení. Že je tedy také serverem je pouze shoda náhod.} a obsluhuje se \glsname{HTTP} požadavky, je tak dosaženo větší univerzálnosti. Například pro aktualizaci osob spjatých s fakultou je třeba odeslat požadavek \texttt{GET} na \glsname{URL} \texttt{http://\linebreak[0]localhost:1337/\linebreak[0]osoby/}, na které je server spuštěný. Zpracování takového požadavku je vyřízeno následujícím kódem ze \textit{spouštěcího} souboru aplikace:
\begin{verbatim}
var express = require('express'); //načtení frameworku express

var app = express.createServer(); //vytvoření serveru

app.get('/osoby/:login?', function (req, res) {
    require('./sources/usermap.js').process(req.params.login);
    res.send("OK: Source enqueued for processing.");
});

//další zdroje zpracovávány obdobně

app.get('*', function (req, res) { //neodchycené cesty
    res.send("Error: Service not found.", 404);
});

app.listen(1337); //spuštění aplikace na portu 1337
\end{verbatim}
V těchto několika řádcích je vytvořený celý server zpracovávající příchozí požadavky. Na pátém řádku je řečeno, že cesta směřující do \textit{segmentu} osoby bude touto metodou odchycena. Odchycen bude i další, nepovinný, segment, který bude uložený do proměnné \texttt{login}. Je tedy možné volat i \glsname{URL} \texttt{http://\linebreak[0]localhost:1337/\linebreak[0]osoby/\linebreak[0]molnaja2/}, aniž by došlo k odchycení dvanáctým řádkem, který klientovi vrátí patřičnou chybu.

Uvnitř metody je na prvním řádku vyžadováno zavolání metody \texttt{process()} z uvedeného souboru s parametrem \texttt{login}. V této metodě je na základě přítomnosti parametru rozhodnuto, jestli se mají zpracovat všechny osoby nebo jen jedna konkrétní. Na druhém řádku je klientovi odesláno oznámení o úspěšném přijetí požadavku.

\subsubsection{Nastavení časovače}
Vzhledem k výše uvedené obsluze \textit{serverové aplikace} \glsname{HTTP} požadavky je možné časovač reprezentovaný cronem jednoduchým nastavením, například:
\begin{verbatim}
# aktualizace jidelnicku kazdy den 3 minuty po zverejneni
3 9,10,16,17 * * * user curl localhost:1337/jidelnicky/\
               2>&1 1> jidelnicky.all.log 2> jidelnicky.err.log

# aktualizace osob kazdy den 10 minut po druhe hodine
10 2 * * * user curl localhost:1337/osoby/\
               2>&1 1> osoby.all.log 2> osoby.err.log

# aktualizace akci 10 minut po kazde sude hodine
10 */2 * * * user curl localhost:1337/akce/\
               2>&1 1> akce.all.log 2> akce.err.log

# aktualizace rozvrhu kazdy den 10 minut po druhe hodine
10 2 * * * user curl localhost:1337/rozvrh/\
               2>&1 1> rozvrh.all.log 2> rozvrh.err.log
# aktualizace rozvrhu v dobe tvorby 10 minut po kazde hodine
10 * * 6,9,10,1,2 * user curl localhost:1337/rozvrh/\
               2>&1 1> rozvrh.all.log 2> rozvrh.err.log
\end{verbatim}
Prvních pět parametrů udává čas vykonání akce (minuta, hodina, den měsíce, měsíc, den týdne), šestý uživatele, zbytek je vykonání příkazu -- odeslání požadavku a zalogování výsledku.

% \subsection{Implementace bezpečnosti}

\subsection{Komplikace}

\subsubsection{Převod kódování}
Jiná kódování v node.js

\subsubsection{Špatně formované XML}
\verb|<literal>>| - https://github.com/antoniogarrote/rdfstore-js/issues/26

% \subsection{Ideální implementace}


\section{Mobilní aplikace}
Pod pojmem \textit{mobilní aplikace} se skrývá aplikace pevně nezávislá na výše uvedené \textit{serverové}. Jejím hlavním cílem je sloužit studentovi \glsname{FIT} \glsname{CVUT} jako průvodce, zároveň má ale plnit i účel ukázky využití \glsname{SPARQL} endpointu vytvořeného v \textit{serverové} části.


\subsection{Řešené problémy}
\todo{Optimalizace, ...}

\subsubsection{Vyhledávání}
\todo{Odstranění diakritiky, regolární výrazy + nezávislost na velikosti písmen.}

\subsubsection{Změna velikosti displeje}
chrome css v px

\subsubsection{Chyby prohlížečů v implementaci specifikací}
Při práci jsem několikrát narazil na situaci, kdy vše v jednom prohlížeči fungovalo, ve druhém nikoliv, a přesto bylo vše implementované správně podle specifikace. V takovém případě pak bývala chyba na straně prohlížeče, který špatně implementoval určitou specifikaci. Situace nastávala převážně u hodně specifických případů. Příkladem může být chybějící podpora pro nastavení transformace \glsname{SVG} elementu JavaScriptem v prohlížeči Google Chrome \cite{BugChromeTran}, tento problém ale lze obejít vlastním sestavením hodnoty celého atributu. 

\subsubsection{Rozdílné DOM u HTML a SVG}
\glsname{SVG} je, co se týče návrhu \glsname{DOM}, komplikovanější, než HTML -- zatímco u HTML stačí pro přístup k hodnotě atributu 

\subsubsection{Zpracování Cross-Origin zdrojů}
\label{sec:mobil:cross-origin}
Cross-Origin Resource Sharing - neumožňuje lézt na cizí server bez ptaní.
mobil/js/lib/jquery.xdomainajax.js, 

\subsubsection{Detekce kódování}
mb\_detect\_encoding - detekce funguje na základě nevalidity - jiné validní prioritnější kódování má přednost.

\subsubsection{Neexistující DNS server}
DNS - mezi DNS servery byl jako zaloha jeden ze serveru provozovany VIC a zda se, ze uz provozovan neni

\subsubsection{Dynamické načítání lokálních souborů}
Z bezpečnostních důvodů došlo v minulosti v prohlížečích k zamezení načítání JavaScriptových souborů z jiných zdrojů, než ze kterých stránka pochází. To v některých případech\footnote{Příkladem je Google Chrome.} přerostlo v zamezení dynamického načítání jakýchkoliv lokálních souborů -- jejich zdroj nemá žádnou hodnotu (\texttt{null}), takže je nelze načíst. Aby byla zachována možnost offline využití aplikace na větší skupině zařízení, byla aplikace přepsána na jediné možné řešení -- explicitní uvedení všech potenciálně vkládaných souborů, což značně omezilo modularitu aplikace.

\todo{inline, , ajax}\verb|<script>|

\todo{Mapa canvas x SVG; vložení SVG: JS string, img, element, object, inline, ajax...}

\subsubsection{Otevírání lokálních souborů}
Omezení zmíněná v předchozím odstavci stupňuje až do extrémů mobilní prohlížeč Chrome Beta, který nepovolí ani přímé otevření lokálního souboru. \todo{Ověřit, možná nefunguje celé Chrome\dots}

\subsection{Implementace bezpečnosti}
\todo{Proxy. jQuery...}



\subsection{Ideální implementace}
Finální implementace se od té ideální, co se týče omezení vzniklých neideálním světem, moc neliší. Vzhledem k cílení aplikace na současná nová zařízení bylo možné realizovat téměř vše, co bylo naplánováno -- v uplynulých letech bylo odvedeno mnoho práce na jednotných standardech napříč zařízeními, a co víc, tyto standardy byly implementovány a z velké části se i dodržují. Problémy se proto vyskytly spíše na poli stále ještě v některých ohledech omezených možnostech mobilních zařízení.

\subsubsection{Malá velikost displeje}
Ačkoliv je toto omezení u mobilních zařízení zřejmé, nic to nemění na faktu, že se s ním musí neustále počítat a je třeba vybírat vhodné komponenty a ty vhodně rozvrhovat. Mým cílem proto bylo vizualizovat jen to nejnutnější.

\subsubsection{Různé typy polohovacích zařízení}
Různá zařízení poskytují různé ovládací prvky využívající různé principy, aplikaci je ale nutné přizpůsobit pro všechny najednou. Jiné je ovládání pro myš (touchpad, trackpad), jiné pro klávesy a jiné pro dotyková zařízení. I přes malý displej je třeba u mobilních dotykových zařízení vytvářet ovládací prvky natolik veliké, aby bylo možné je používat prsty.

\subsubsection{Optimalizace rozvržení prvků}
Mobilní zařízení nabízejí pro lepší využití malých displejů mnoho různých otimalizací rozvržení zobrazovaných prvků. Zpravidla jsou tyto vlastnosti velmi užitečné -- komprimují nevyužité místo, ve specifických případech, jako je třeba přesné pozicování ukazatele, ale může docházet k obtížně řešitelným situacím. Jediné místo aplikace, které se bez přesného rozmístění neobejde, jsou mapy. Problém jsem se nakonec rozhodl řešit využitím  \glsname{SVG} mapových podkladů, ve kterých prohlížeče, zdá se, zatím rozvržení neoptimalizují.

\subsubsection{Implementace SVG}
Implementace \glsname{SVG} je v desktopových prohlížečích na velmi dobré úrovni již nějakou dobu,\footnote{Posledním významným desktopovým prohlížečem podporujícím \glsname{SVG} se stal s dlouhým odstupem verzí 9.0 Internet Explorer \cite{CanIUse}, předtím pro něj ale byly dostupné vhodné pluginy.} v nedávné minulosti se ale podpora značně zlepšila i na poli mobilních zařízení \cite{CanIUse}.

U mobilních zařízení nepřekvapuje chybějící implementace některých neklíčových možností \glsname{SVG}, jako je třeba průhlednost v případě Opery Mobile. V této oblasti v současné době rozsáhlému nasazení nepřeje \glsname{SVG} nepodporující hojně využívaný nativní webový prohlížeč Androidu verze 2, od verze 3 je podpora zavedena. Problém lze obejít využitím jiného dostupného prohlížeče.

Překvapením byla pomalá, ačkoliv jinak velmi rozsáhlá, implementace \glsname{SVG} v prohlížeči Mozilla Firefox -- po některých operacích, v mém případě to byla třeba změna hodnot atributu \texttt{viewBox} \glsname{SVG} elementu (posunutí mapy), prohlížeč vyžaduje překreslení celého obrázku, což je velmi zdlouhavé a zabraňuje to plynulému posunu mapy z aktuální na novou pozici, byť by to bylo vhodné pro lepší uživatelovo zorientování se. Tento nedostatek je známý již delší dobu \cite{Bugzilla}.

\subsubsection{Absence popisků}
Velmi užitečnou vlastností dostupnou u desktopových aplikací jsou popisky zobrazující se po najetí myši nad podporovaný prvek. Dá se tímto způsobem velmi dobře implementovat kontextová nápověda, která patří mezi nejpodstatnější zdroje pomoci uživateli. U dotykových mobilních zařízení ale bohužel popisky zpravidla zobrazovat nejdou -- najetím, tedy dotykem prstu, se rovnou vyvolá akce prvku, aniž by se popisek mohl zobrazit.

